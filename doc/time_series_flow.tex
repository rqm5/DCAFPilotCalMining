% \documentclass{article}
%% \usepackage[latin1]{inputenc}
%% \usepackage{tikz}
%% \usetikzlibrary{shapes,arrows}
%% \usepackage{caption}
%% \newcommand*{\h}{\hspace{5pt}}% for indentation
%% \newcommand*{\hh}{\h\h}% double indentation
%% \begin{document}
\begin{figure}
\begin{center}
  % setting the typeface to sans serif and the font size to small
  % the scope local to the environment
  \sffamily
  \footnotesize
  \begin{tikzpicture}[auto, 
      %decision/.style={diamond, draw=black, thick, fill=white,
      %text width=8em, text badly centered,
      %inner sep=1pt, font=\sffamily\small},
      block_center/.style ={rectangle, draw=black, thick, fill=white,
        text width=10em, text centered,
        minimum height=4em},
      block_left/.style ={rectangle, draw=black, thick, fill=white,
        text width=16em, text ragged, minimum height=4em, inner sep=6pt},
      block_noborder/.style ={rectangle, draw=none, thick, fill=none,
        text width=12em, text centered, minimum height=1em},
      block_assign/.style ={rectangle, draw=black, thick, fill=white,
        text width=18em, text ragged, minimum height=3em, inner sep=6pt},
      block_lost/.style ={rectangle, draw=black, thick, fill=white,
        text width=16em, text ragged, minimum height=3em, inner sep=6pt},
      line/.style ={draw, thick, -latex', shorten >=0pt}]
    %
    % 1. outlining the flowchart using the PGF/TikZ matrix funtion
    \matrix [column sep=5mm,row sep=3mm] {
      % enrollment - row 1
      \node [block_center] (w1) {Dataset access records in week 1};
      & \node [block_center] (w2) {Dataset access records in week 2};
      & \node [block_center] (cf_dp) {Conference dump from database + schema}; 
      \\
      % enrollment - row 1'
      \node [block_noborder] (ts1) {\verb|time_series.py|};
      &
      & \node [block_noborder] (cf_parser) {\verb|cms_conf_parser.py|}; \\
      % enrollment - row 2
      \node [block_center] (ts2) {A dataset's weekly access count time series};
      &
      & \node [block_center] (cf_ct) {Weekly conference count time series}; \\
      % enrollment - row 2'
      \node [block_noborder] (ts31) {\verb|time_series.py|}; 
      & \node [block_noborder] (ts32) {\verb|time_series.py|}; 
      & \node [block_noborder] (ts33) {\verb|time_series.py|}; \\
      % enrollment - row 3
      \node [block_center] (result1) {Seasonality};
      & \node [block_center] (result2) {Cross correlation};
      & \node [block_center] (result3) {Seasonality}; \\
    };% end matrix
    %
    %
    % 2. connecting nodes with paths
    \begin{scope}[every path/.style=line]
      % paths for enrollemnt rows
      \path (w1)   -- (ts1);
      \path (w2)   -- (ts1);
      \path(ts1) -- (ts2);
      \path (cf_dp) -- (cf_parser);
      \path (cf_parser) -- (cf_ct);
      % 
      \path (ts2) -- (ts32); 
      \path (cf_ct) -- (ts32);
      \path (ts32) -- (result2);      
      \path (ts2) -- (ts31); 
      \path (ts31) -- (result1);
      \path (cf_ct) -- (ts33); 
      \path (ts33) -- (result3);
    \end{scope}
\end{tikzpicture}
\captionof{figure}{Flowchart of Time Series Generation and Analysis}
\label{chart1}
\end{center}
\end{figure}

% \end{document}
